\documentclass[a4paper]{jsarticle}
\usepackage{iapaper}
\usepackage[dvipdfmx]{graphicx}
\setcounter{tocdepth}{3} %subsubsectionを表示するために

\begin{document}
% 修士論文の場合は \degreethesis を使わず,下記を使う.
% なお博士の場合は \doctorthesisにすること
 \masterthesis
% 卒業論文の場合は下記を使う
% \degreethesis

\title{修論のテーマ}
\date{平成00年度}
\advisor{串山久美子} % 指導教員名を入れる
\IDnumber{17893528} 
\Mauthor{宮下 恵太} % 自分の名前を入れる.修士の場合は \Mauthor{氏名} に変更する.
\submissiondate{平成00年0月00日}
\maketitle

\pagenumbering{roman} % 要旨はRoman書体で表示
\setcounter{page}{1} % 1から振り直す
\jasummary{和文要旨}{}\\
\thispagestyle{plain}
 アイウエオ、カキクケコ。

\summary{英文要旨}{} 
\thispagestyle{plain}
abcde, fghijk.

\newpage

\makemokuji %目次を自動で出力

\newpage %新しいページを作る時はこの文書を使ってください

\pagenumbering{arabic}  % 論文本体はArabicで表示
\setcounter{page}{1} % ページ番号を1から振り直す

\section{序論}
\subsection{はじめに}
---
\section{研究背景}
\subsection{可聴化}
ソニフィケーション!

\subsection{サウンドアート}
サウンドアート!
\section{制作・実装}
\subsection{作品名1}
\subsection{作品名2}
\section{結論}

\newpage

\begin{thebibliography}{999}

\bibitem{参考文献}
著者名, タイトル, 学会誌のページ数, 公開時間


\end{thebibliography}

\end{document}



%\section*{論文の提出期限}
%印刷論文原稿を3部, 1月28日(金)に大学院教務担当教員へ提出すること. 